\chapter{Introduction}\label{ch:1}

\section{Background of the Study}\label{sec:1-bos}

Engineering education is the process of acquiring skills, knowledge, and the competency needed for the preparation of an individual to become a professional engineer. Mechanics is a branch of physics that deals with the effects of forces and energy on bodies. Engineering Mechanics is divided into two major parts, Statics and Dynamics. One of the foundational subjects in the engineering curriculum is Statics of Rigid Bodies. Statics of Rigid Bodies is a branch of engineering mechanics that studies how forces and moments act on particles and rigid bodies at rest or in equilibrium. 

In the Civil Engineering curriculum of Mindanao State University - General Santos (MSU-Gensan), as stated in the MSU Board Resolution No. 375, s. 2017, Statics of Rigid Bodies (ENS161) serves as one of the core foundation courses in Engineering Mechanics. It is a prerequisite for Dynamics of Rigid Bodies (ENS162) and Mechanics of Deformable bodies (CVE155), both of which are essential for advanced engineering studies. In other engineering programs, Statics of Rigid Bodies also functions as a preparatory course for more advanced subjects in mechanical, electrical and electronics engineering. This highlights the significance of Statics of Rigid Bodies as a subject that cultivates the analytical and problem-solving skills fundamental to engineering education. Failure in this subject can delay a student’s academic progress since it serves as a prerequisite for higher-level engineering courses. 

In Academic Year 2025-2026, most students taking Statics of Rigid Bodies are second-year engineering students, typically aged 19 to 20 years old. This generation, often referred to as Generation Z and Digital Natives, has grown up with access to digital technology and online learning resources. As digital natives, they consider technology to be an integral and necessary part of their lives. They are generally more engaged with interactive, web-based learning environments compared to traditional classroom methods.  

Students often encounter several difficulties in learning Statics of Rigid Bodies. One common problem is their failure to connect mathematical symbols with their corresponding physical meanings, as they tend to view forces as mere numerical values or abstract mathematical entities rather than understanding them as interactions between bodies. Many also struggle with constructing accurate Free Body Diagrams (FBDs), often drawing forces incorrectly, which reflects weak conceptual understanding. Additionally, a poor understanding of equilibrium is also evident, as some learners attempt to solve problems before fully mastering the basic ideas of force perception and static equivalence. Moreover, many students have trouble visualizing the forces acting between inanimate objects, such as a book resting on a table, revealing gaps in intuitive understanding. These challenges are often compounded by an overreliance on memorization and mathematical manipulation, where students focus more on formulas than on conceptual reasoning. Lastly, the lack of engagement and interaction in traditional lecture-based teaching limits opportunities for active learning, collaboration, and the development of deeper comprehension through visual or hands-on experiences.

To address the conceptual and computational challenges encountered by engineering students in learning Statics of Rigid Bodies, the researchers developed an interactive web-based learning platform named StatiCalcs. The platform integrates multiple statics calculators capable of solving a wide range of engineering problems while providing step-by-step solutions and automatically generated free-body diagrams. These features allow users to visualize the forces acting on bodies and to gain a deeper understanding of equilibrium conditions. The website is systematically organized according to the major topics covered in Statics, including Introduction to Statics, Force Systems, Equilibrium, Structures, and Distributed Loads. By combining computation, visualization, and guided explanation, StatiCalcs serves as a supplementary learning tool designed to enhance students’ comprehension and problem-solving abilities. Furthermore, it provides a more interactive and technology-driven learning environment that aligns with modern approaches to engineering education and promotes deeper engagement with fundamental concepts.

\section{Statement of the Problem}\label{sec:1-sop}

This study aims to determine the level of perception of Civil Engineering students and instructors of MSU-General Santos on the developed web-based learning tool, \textit{StatiCalc}, designed for the subject \textit{Statics of Rigid Bodies}. The study seeks to assess how the tool performs in terms of usability, accessibility, and user satisfaction.

Specifically, this study aims to answer the following questions:

\begin{enumerate}
    \item What is the level of perception of Civil Engineering students of MSU-General Santos regarding \textit{StatiCalc} in terms of:
    \begin{enumerate}[label*=\alph*.]
        \item Usability
        \item Accessibility
        \item Satisfaction
    \end{enumerate}

    \item What is the level of perception of Civil Engineering instructors of MSU-General Santos regarding \textit{StatiCalc} in terms of:
    \begin{enumerate}[label*=\alph*.]
        \item Usability
        \item Accessibility
        \item Satisfaction
    \end{enumerate}

    \item Is there a significant difference between the level of perception of students and instructors in terms of usability, accessibility, and satisfaction?
\end{enumerate}

\section{Scope and Limitations}\label{sec:1-sal}
This study focuses on the design, development, and evaluation of a web-based learning tool, StatiCalc, intended to assist Civil Engineering students of Mindanao State University – General Santos in studying the subject Statics of Rigid Bodies. The website covers all major topics in Statics, including force systems, equilibrium, structures, centroids, moments of inertia, and friction. It integrates conceptual explanations and calculators to help users understand and solve problems interactively. The evaluation of the tool focuses on three aspects: usability, accessibility, and user satisfaction.

The respondents of the study are limited to Civil Engineering students, specifically second-year students currently enrolled in Statics of Rigid Bodies and senior students who have already taken the subject, as well as instructors teaching the same course in the College of Engineering of MSU-General Santos during the Academic Year 2025–2026. The data were collected through survey questionnaires administered after the respondents had explored and used the developed website

The study does not measure students’ academic performance or actual improvement in examination results. It also excludes students and instructors from other engineering programs and subjects such as Dynamics and Mechanics of Materials. Moreover, the study focuses solely on perceptions regarding the usability and effectiveness of the developed tool and does not account for external factors such as internet connectivity, learning motivation, or prior knowledge of the respondents.

\section{Significance of the Study}\label{sec:1-sots}

This study is conducted to develop a web-based learning tool, StatiCalc, which aims to support the teaching and learning of Statics of Rigid Bodies among Civil Engineering students of MSU-General Santos. The results and outcomes of this research are expected to benefit the following:
Students. The developed website serves as a supplementary learning tool that allows students to practice problem-solving in Statics of Rigid Bodies through an interactive and user-friendly platform. As learners today are inclined toward technology-based education, StatiCalc provides an accessible and engaging way to enhance understanding and reinforce classroom instruction.
Teachers. The tool can be integrated into teaching strategies to improve the delivery of complex statics concepts. It may also assist instructors in providing visual and computational demonstrations that complement lectures and classroom exercises.
School Administration. The study may provide the institution with an example of how digital learning tools can be developed and implemented to improve academic performance. It may also encourage future initiatives that promote technology-driven learning in other engineering courses.
Future Researchers. This study may serve as a reference for future studies that aim to develop similar educational tools or further improve StatiCalc by adding new features, expanding its scope, or evaluating its long-term impact on student learning.


\section{Conceptual Framework}\label{sec:1-cf}

\section{Definition of Terms}\label{sec:1-dot}

\vspace*{-2em}
\begin{longtblr}[
	entry=none,
	label=none
	]{rowsep=2mm, colspec={t{5cm,l}t{8.5cm,l}}}

\textbf{Statics of Rigid Bodies} & 
Statics of Rigid Bodies is a branch of engineering mechanics that deals with the study of forces and their effects on bodies that are assumed to remain perfectly rigid. It focuses on determining the conditions of equilibrium, where the sum of all forces and moments acting on a body equals zero to ensure structural stability. \\

\textbf{Equilibrium} &
Equilibrium refers to the condition of a body when the sum of all forces and the sum of all moments acting upon it are equal to zero, meaning the body remains at rest or moves with constant velocity. \\

\textbf{Free Body Diagram (FBD)} &
A Free Body Diagram (FBD) is a simplified graphical representation of a body or system isolated from its surroundings, showing all external forces and moments acting on it. It is used to visualize and analyze the forces acting on the body. \\

\textbf{Web-Based Learning Tool} &
A web-based learning tool is an interactive educational platform or software accessible through the internet, designed to facilitate learning through engagement, visualization, and feedback. \\

\textbf{Developmental Research} &
Developmental research is a methodological approach that focuses on the systematic design, development, and evaluation of instructional programs, processes, or products to improve educational practice. \\

\textbf{Perception} &
Perception is the process by which individuals interpret and organize sensory information to create meaning and understanding of their environment. In educational research, it often refers to how learners view or experience a certain tool, concept, or environment. \\

\textbf{Usability} &
Usability refers to the degree to which a product, tool, or interface allows users to achieve specific goals effectively, efficiently, and satisfactorily in a defined context. \\

\textbf{Accessibility} &
Accessibility is the design of products, systems, or environments that can be used by people of all abilities, including those with disabilities, to perceive, understand, navigate, and interact effectively. \\

\textbf{Satisfaction} &
Satisfaction refers to the degree to which users feel content or fulfilled with a product, service, or experience, often reflecting how well it meets their expectations and needs. \\

\textbf{Supplementary} &
Supplementary refers to something that is added to complete, enhance, or support something else. In the context of education or research, supplementary materials or tools provide additional resources that aid in understanding, reinforcement, or enrichment of learning. \\

\textbf{Digital Natives} &
Digital natives are individuals who have grown up during the age of digital technology, such as computers, smartphones, and the internet, and are thus comfortable using these tools in daily life. \\

\textbf{Generation Z} &
Generation Z refers to the demographic cohort born roughly between the mid-to-late 1990s and the early 2010s. They are characterized by their familiarity with digital technology, internet connectivity, and social media from an early age. Generation Z is often described as tech-savvy, socially aware, and highly engaged with online communication and learning environments. \\

\end{longtblr}


