\chapter{Introduction}\label{ch:1}

\section{Background of the Study}\label{sec:1-bos}

Engineering education is the process of acquiring the knowledge, skills, and competencies necessary to prepare individuals for professional engineering practice. One of its foundational areas is \textit{Engineering Mechanics}, which concerns the effects of forces and energy on bodies. It is divided into two branches: \textit{Statics} and \textit{Dynamics}. Among these, \textit{Statics of Rigid Bodies} plays a crucial role in the engineering curriculum because it establishes the principles of equilibrium and force analysis, which are required in advanced engineering courses. It focuses on analyzing how forces and moments act on particles and rigid bodies that remain at rest or in equilibrium. 

In the Civil Engineering curriculum of Mindanao State University – General Santos (MSU-Gensan), as stated in \textit{MSU Board Resolution No. 375, s. 2017}, \textit{Statics of Rigid Bodies (ENS 161)} serves as one of the core foundation courses in Engineering Mechanics. It is a prerequisite for \textit{Dynamics of Rigid Bodies (ENS 162)} and \textit{Mechanics of Deformable Bodies (CVE 155)}, both of which are essential for higher-level design and analysis. In other engineering programs, such as Mechanical, Electrical, and Agricultural and Biosystems Engineering, Statics also functions as a preparatory course that cultivates analytical and problem-solving skills fundamental to engineering education. Failure in this subject often results in academic delays, as it serves as a prerequisite for subsequent major engineering courses. \autocite{salami_challenges_2025}

Despite its foundational importance, \textit{Statics of Rigid Bodies} remains one of the most conceptually and mathematically demanding subjects in the engineering curriculum. The discipline requires a solid grasp of trigonometry, geometry, and algebra. Without this mathematical foundation, students struggle to analyze forces, moments, and equilibrium conditions accurately. 

A study conducted at Baguio Central University revealed that incoming first-year students exhibited mathematical difficulties that affected their readiness for college-level subjects, particularly in algebra \autocite{felix_analyzing_2023}. Similarly, research among first-year students at Mandaue City College found that math anxiety, a feeling of tension and apprehension when solving mathematical problems, negatively correlates with academic performance, meaning that higher levels of math anxiety are associated with lower academic achievement. \autocite{incierto_mathematics_nodate}

The use of calculators and digital tools has been shown to help reduce computational errors and mitigate math anxiety by allowing students to focus on conceptual understanding rather than manual computation. \autocite{segarra_does_2024} Studies further indicate that combining calculator use with structured instruction enhances problem-solving confidence and overall comprehension.

In addition to mathematical difficulties, students face several conceptual and procedural challenges in learning Statics of Rigid Bodies. Many struggle to connect physical concepts with their mathematical representations, often perceiving forces and moments as abstract quantities rather than as real interactions between bodies. This lack of conceptual understanding frequently leads to errors in constructing Free Body Diagrams (FBDs), such as omitting essential forces, misplacing directions, or confusing internal and external forces. Learners also experience difficulty distinguishing between related concepts such as moments, couples, and resultant forces, as well as visualizing how forces act within two- and three-dimensional structures. Moreover, students often rely on rote memorization of formulas and procedural steps without fully grasping their physical meaning, resulting in shallow learning and difficulty applying knowledge to new problem situations. These difficulties are further compounded by traditional lecture-based teaching methods, which limit interaction, visualization, and engagement, factors essential for developing a deeper understanding of statics principles \autocite{salami_challenges_2025}. 

Students currently taking \textit{Statics of Rigid Bodies} at MSU-Gensan are generally second-year Civil Engineering students, typically aged 19 to 20. This group belongs to Generation Z, often referred to as digital natives, individuals who have grown up surrounded by computers, smartphones, and the internet. According to \cite{szymkowiak_information_2021}, educators should integrate modern, Internet-based learning tools, such as mobile applications and online videos, alongside traditional instruction to align with this generation's learning preferences. Similarly, \cite{zeichner_using_2020} emphasized that learning through simulations fosters a deeper understanding of abstract concepts than conventional teaching methods. Consistent with the findings of \cite{de_la_hoz_self-explanation_2023}, traditional lecture-based approaches in statics are insufficient to address these learning barriers, highlighting the need for innovative digital tools that enhance engagement, comprehension, and problem-solving skills among engineering students. In the context of \textit{Statics of Rigid Bodies}, where students often struggle with visualization and conceptualization, these insights underscore the need for interactive, technology-supported learning environments.

Given these mathematical and conceptual challenges, there is a clear need for a supplementary, technology-driven learning resource that can combine computational assistance, visualization, and interactivity. To address this need, the researchers developed \textit{StatiCalcs}, an interactive web-based learning tool designed specifically for \textit{Statics of Rigid Bodies}. The platform organizes topics according to the major chapters of the course: Introduction to Statics, Force Systems, Equilibrium, Structures, and Distributed Loads. And it integrates calculators that generate step-by-step solutions, results, and free-body diagrams in real time. By merging conceptual explanations with computation and visualization, \textit{StatiCalc} aims to enhance students’ understanding, reduce learning anxiety, and reinforce classroom instruction.

Recognizing students' characteristics, the researchers saw potential in developing a digital learning platform that aligns with students’ familiarity with technology while addressing learning challenges in \textit{Statics of Rigid Bodies}. Through this innovation, the study seeks to bridge the gap between traditional teaching methods and digital learning practices by providing MSU-Gensan Civil Engineering students with an accessible, interactive, and effective academic support tool for studying \textit{Statics of Rigid Bodies}.

\section{Statement of the Problem}\label{sec:1-sop}

This study aims to determine the level of perception of Civil Engineering students and instructors of MSU-General Santos on the developed web-based learning tool, \textit{StatiCalc}, designed for the subject \textit{Statics of Rigid Bodies}. The study seeks to assess how the tool performs in terms of usability, accessibility, and user satisfaction.

Specifically, this study aims to answer the following questions:

\begin{enumerate}
    \item What is the level of perception of Civil Engineering students of MSU-General Santos regarding \textit{StatiCalc} in terms of:
    \begin{enumerate}[label*=\alph*.]
        \item Usability
        \item Accessibility
        \item Satisfaction
    \end{enumerate}

    \item What is the level of perception of Civil Engineering instructors of MSU-General Santos regarding \textit{StatiCalc} in terms of:
    \begin{enumerate}[label*=\alph*.]
        \item Usability
        \item Accessibility
        \item Satisfaction
    \end{enumerate}

    \item Is there a significant difference between the level of perception of students and instructors in terms of usability, accessibility, and satisfaction?
\end{enumerate}

\section{Scope and Limitations}\label{sec:1-sal}
This study focuses on the design, development, and evaluation of a web-based learning tool, StatiCalc, intended to assist Civil Engineering students of Mindanao State University – General Santos in studying the subject Statics of Rigid Bodies. The website covers all major topics in Statics, including force systems, equilibrium, structures, centroids, moments of inertia, and friction. It integrates conceptual explanations and calculators to help users understand and solve problems interactively. The evaluation of the tool focuses on three aspects: usability, accessibility, and user satisfaction.

The respondents of the study are limited to Civil Engineering students, specifically second-year students currently enrolled in Statics of Rigid Bodies and senior students who have already taken the subject, as well as instructors teaching the same course in the College of Engineering of MSU-General Santos during the Academic Year 2025–2026. The data were collected through survey questionnaires administered after the respondents had explored and used the developed website

The study does not measure students’ academic performance or actual improvement in examination results. It also excludes students and instructors from other engineering programs and subjects such as Dynamics and Mechanics of Materials. Moreover, the study focuses solely on perceptions regarding the usability and effectiveness of the developed tool and does not account for external factors such as internet connectivity, learning motivation, or prior knowledge of the respondents.

\section{Significance of the Study}\label{sec:1-sots}

This study is conducted to develop a web-based learning tool, StatiCalc, which aims to support the teaching and learning of Statics of Rigid Bodies among Civil Engineering students of MSU-General Santos. The results and outcomes of this research are expected to benefit the following:
Students. The developed website serves as a supplementary learning tool that allows students to practice problem-solving in Statics of Rigid Bodies through an interactive and user-friendly platform. As learners today are inclined toward technology-based education, StatiCalc provides an accessible and engaging way to enhance understanding and reinforce classroom instruction.
Teachers. The tool can be integrated into teaching strategies to improve the delivery of complex statics concepts. It may also assist instructors in providing visual and computational demonstrations that complement lectures and classroom exercises.
School Administration. The study may provide the institution with an example of how digital learning tools can be developed and implemented to improve academic performance. It may also encourage future initiatives that promote technology-driven learning in other engineering courses.
Future Researchers. This study may serve as a reference for future studies that aim to develop similar educational tools or further improve StatiCalc by adding new features, expanding its scope, or evaluating its long-term impact on student learning.


\section{Conceptual Framework}\label{sec:1-cf}

\section{Definition of Terms}\label{sec:1-dot}

\vspace*{-2em}
\begin{longtblr}[
	entry=none,
	label=none
	]{rowsep=2mm, colspec={t{5cm,l}t{8.5cm,l}}}

\textbf{Statics of Rigid Bodies} & 
Statics of Rigid Bodies is a branch of engineering mechanics that deals with the study of forces and their effects on bodies that are assumed to remain perfectly rigid. It focuses on determining the conditions of equilibrium, where the sum of all forces and moments acting on a body equals zero to ensure structural stability. \\

\textbf{Equilibrium} &
Equilibrium refers to the condition of a body when the sum of all forces and the sum of all moments acting upon it are equal to zero, meaning the body remains at rest or moves with constant velocity. \\

\textbf{Free Body Diagram (FBD)} &
A Free Body Diagram (FBD) is a simplified graphical representation of a body or system isolated from its surroundings, showing all external forces and moments acting on it. It is used to visualize and analyze the forces acting on the body. \\

\textbf{Web-Based Learning Tool} &
A web-based learning tool is an interactive educational platform or software accessible through the internet, designed to facilitate learning through engagement, visualization, and feedback. \\

\textbf{Developmental Research} &
Developmental research is a methodological approach that focuses on the systematic design, development, and evaluation of instructional programs, processes, or products to improve educational practice. \\

\textbf{Perception} &
Perception is the process by which individuals interpret and organize sensory information to create meaning and understanding of their environment. In educational research, it often refers to how learners view or experience a certain tool, concept, or environment. \\

\textbf{Usability} &
Usability refers to the degree to which a product, tool, or interface allows users to achieve specific goals effectively, efficiently, and satisfactorily in a defined context. \\

\textbf{Accessibility} &
Accessibility is the design of products, systems, or environments that can be used by people of all abilities, including those with disabilities, to perceive, understand, navigate, and interact effectively. \\

\textbf{Satisfaction} &
Satisfaction refers to the degree to which users feel content or fulfilled with a product, service, or experience, often reflecting how well it meets their expectations and needs. \\

\textbf{Supplementary} &
Supplementary refers to something that is added to complete, enhance, or support something else. In the context of education or research, supplementary materials or tools provide additional resources that aid in understanding, reinforcement, or enrichment of learning. \\

\textbf{Digital Natives} &
Digital natives are individuals who have grown up during the age of digital technology, such as computers, smartphones, and the internet, and are thus comfortable using these tools in daily life. \\

\textbf{Generation Z} &
Generation Z refers to the demographic cohort born roughly between the mid-to-late 1990s and the early 2010s. They are characterized by their familiarity with digital technology, internet connectivity, and social media from an early age. Generation Z is often described as tech-savvy, socially aware, and highly engaged with online communication and learning environments. \\

\end{longtblr}


