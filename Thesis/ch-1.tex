
\justifying % fully justified text

\chapter{Introduction}\label{ch:1}

\section{Background of the Study}\label{sec:1-bos}

Engineering Mechanics is one of the fundamental components of engineering education, and Statics of Rigid Bodies serves as its essential foundation. The course develops the analytical skills needed to understand how forces and moments act on bodies at rest, making it a prerequisite for more advanced engineering subjects. In the Civil Engineering curriculum of Mindanao State University -- General Santos (MSU--Gensan), \textit{Statics of Rigid Bodies (ENS161)} is a core subject required for academic progression in the engineering program.

Despite its importance, many students find Statics difficult due to its mathematical requirements and abstract nature. Students often struggle with force systems, moments, and the construction of accurate \textit{Free Body Diagrams (FBDs)}, which are necessary for solving Statics of Rigid Bodies problems. \cite{salami_challenges_2025} found that learners commonly face conceptual and procedural difficulties in Statics, including challenges in visualizing forces and connecting equations to physical situations. These difficulties are further intensified by weak mathematical foundations, particularly in algebra and trigonometry, which are essential for problem-solving in Statics. \cite{felix_analyzing_2023} noted that incoming college students often exhibit declining mathematical readiness, affecting their ability to perform well in computation-heavy subjects.

Additionally, affective factors such as math anxiety contribute to students’ performance challenges. \cite{incierto_mathematics_nodate} found that higher levels of math anxiety are associated with lower academic achievement, indicating that many students struggle not only with understanding concepts but also with the emotional stress associated with solving mathematical problems. Research also shows that appropriate tools, such as calculators and structured digital aids, can reduce computational errors and alleviate anxiety by allowing learners to focus on conceptual understanding \autocite{segarra_does_2024}.

Traditional lecture-based instruction, which relies heavily on verbal explanation and boardwork, often provides limited visualization and interactivity. Modern learners, particularly those belonging to Generation Z, respond more positively to technology-supported learning environments. \cite{szymkowiak_information_2021} emphasized that digital-native students benefit from online tools, multimedia resources, and interactive platforms that enhance engagement and comprehension.

Given these challenges, students benefit from supplementary learning tools that reinforce classroom instruction, enhance visualization, and support step-by-step problem solving. In response, the study developed \textit{StatiCalcs}, a web-based interactive learning tool designed to help MSU–Gensan Civil Engineering students understand Statics of Rigid Bodies. The platform provides topic-aligned resources and integrated calculators that generate step-by-step solutions and visual representations, offering an accessible and technology-supported supplement to traditional instruction.

%---------------------------------------------------------------------

\section{Statement of the Problem}\label{sec:1-sop}

This study aims to develop a web-based learning tool, \textit{StatiCalcs}, specifically designed for the subject Statics of Rigid Bodies. After the development of the application, the study seeks to determine the level of perception of Civil Engineering students and instructors of MSU--General Santos regarding its usability, accessibility, and user satisfaction. 

\bigskip
Specifically, this study aims to answer the following questions:

\begin{enumerate}
    \item What is the level of perception of Bachelor of Science in Civil Engineering students of MSU--General Santos regarding \textit{StatiCalcs} in terms of:
    \begin{enumerate}
        \renewcommand{\labelenumii}{\theenumi.\arabic{enumii}}
        \item Usability;
        \item Accessibility; and
        \item Satisfaction?
    \end{enumerate}

    \item What is the level of perception of Bachelor of Science in Civil Engineering instructors of MSU--General Santos regarding \textit{StatiCalcs} in terms of:
    \begin{enumerate}
        \renewcommand{\labelenumii}{\theenumi.\arabic{enumii}}
        \item Usability;
        \item Accessibility; and
        \item Satisfaction?
    \end{enumerate}

    \item Is there a significant difference between the level of perception of Bachelor of Science in Civil Engineering students and instructors regarding \textit{StatiCalcs} in terms of usability, accessibility, and satisfaction?
\bigskip

    \textbf{Null Hypothesis}

    \textbf{Ho:} There is no significant difference between the level of perception of Bachelor of Science in Civil Engineering students and instructors regarding \textit{StatiCalcs} in terms of usability, accessibility, and satisfaction.

\end{enumerate}

%---------------------------------------------------------------------

\section{Scope and Limitations}\label{sec:1-sal}

This study will focus on the design, development, and evaluation of a web-based learning tool, \textit{StatiCalcs}, intended to assist Bachelor of Science in Civil Engineering students of Mindanao State University -- General Santos in studying the subject Statics of Rigid Bodies. The website will cover all major topics in Statics, including force systems, equilibrium, structures, centroids, moments of inertia, and friction. It will integrate step-by-step solutions and calculators to help users understand and solve problems interactively. The evaluation of the tool will focus on three aspects: usability, accessibility, and user satisfaction.

The respondents of the study will be limited to Bachelor of Science in Civil Engineering students, specifically second-year students currently enrolled in Statics of Rigid Bodies and students who have already taken the subject, as well as instructors teaching the same course in the College of Engineering of MSU--General Santos during the Academic Year 2025--2026. Data will be collected through survey questionnaires administered after respondents have explored and used the developed website.

The study will not measure students' academic performance or actual improvement in examination results. It will also exclude students and instructors from other engineering programs. Moreover, the study will focus solely on perceptions of the usability and effectiveness of the developed tool. It will not account for external factors such as internet connectivity, learning motivation, or respondents' prior knowledge.

%.....

The scope of this project will include the development of a website using modern web technologies, focusing on both frontend and backend functionality. The backend of the website will be built in Java, handling application logic and processing user interactions. The frontend interface will be designed with Tailwind CSS, ensuring a responsive, modern, and visually consistent user experience. Version control and collaboration will be managed through GitHub, while deployment and hosting will be automated via Netlify. ChatGPT will be utilized as a supplementary tool to assist in web design decisions, code arrangement, and debugging, providing guidance and improving workflow efficiency.

The project’s delimitations and limitations will include reliance on specific technologies and platforms. The backend will be limited to Java, which may restrict integration with other programming languages or frameworks without additional configuration. The frontend design will depend on Tailwind CSS, and deviations from its utility-first approach may require additional custom CSS. While GitHub and Netlify will streamline version control and deployment, they will require internet connectivity and account setup, which could pose challenges in offline or restricted environments. ChatGPT will serves only as a supportive tool and does not replace the developer’s judgment or expertise. Additionally, the website’s functionality and performance are constrained by the features provided by the chosen technologies and hosting platform, limiting scalability for highly complex or resource-intensive applications.




%---------------------------------------------------------------------

\section{Significance of the Study}\label{sec:1-sots}


This study will focus on developing a web-based learning tool, \textit{StatiCalcs}, to support the teaching and learning of Statics of Rigid Bodies among Bachelor of Science in Civil Engineering students at MSU--General Santos. The results and outcomes of this research may benefit the following:

\medskip
\textbf{Students.} The developed website may serve as a supplementary learning tool, allowing students to practice problem-solving in Statics of Rigid Bodies through an interactive, user-friendly platform. As learners today are inclined toward technology-based education, \textit{StatiCalcs} may provide an accessible and engaging way to enhance understanding and reinforce classroom instruction.

\medskip
\textbf{Teachers.} The tool may be integrated into teaching strategies to improve the delivery of complex statics concepts. It may assist instructors in providing visual and computational demonstrations that complement lectures and classroom exercises.

\medskip
\textbf{School Administration.} The study may provide the institution with an example of how to develop and implement digital learning tools to improve academic performance. It may also encourage future initiatives that promote technology-driven learning in other engineering courses.

\medskip
\textbf{Future Researchers.} This study may serve as a reference for future studies aiming to develop similar educational tools or further improve \textit{StatiCalcs} by adding new features, expanding its scope, or evaluating its long-term impact on student learning.

%---------------------------------------------------------------------

\section{Conceptual Framework}\label{sec:1-cf}


\begin{figure}[H]
    \centering
    \includegraphics[width=0.9\textwidth]{assets/Concept.png}
    \caption{Conceptual Framework}
    \label{fig:conceptual_framework}
\end{figure}

This study is anchored in a conceptual framework that describes the relationship between users of the system and the web-based learning tool \textit{StatiCalcs}, and how their interaction with the platform leads to the evaluation of three key system attributes: usability, accessibility, and satisfaction. The independent variables are the students and instructors, while the dependent variables are their level of perception in terms of usability, accessibility, and satisfaction.

\textit{StatiCalcs} functions as the central intervention, offering computational features and step-by-step problem-solving support. The quality of users’ interactions with the system is evaluated through three outcome variables: usability, accessibility, and satisfaction. Usability pertains to the efficiency and ease with which users navigate and perform tasks within the platform. Accessibility evaluates the system’s ability to accommodate diverse users, devices, and learning conditions. Satisfaction measures the overall acceptance of the tool and the extent to which it meets user expectations.

Overall, the framework posits that students' and teachers' experiences with \textit{StatiCalcs} directly shape their perceptions of its usability, accessibility, and satisfaction, thereby informing the system’s effectiveness as a supplementary learning tool.

%---------------------------------------------------------------------

\section{Definition of Terms}\label{sec:1-dot}

%add more terms

\vspace*{-2em}
\begin{longtblr}[
    entry=none,
    label=none
]{rowsep=2mm, colspec={t{5cm,l}t{8.5cm,j}}}

\textbf{Accessibility} &
The extent to which Bachelor of Science in Civil Engineering students enrolled in Statics of Rigid Bodies and their instructors can easily access, navigate, and use the developed website, \textit{StatiCalcs}, regardless of ability, to support effective learning and teaching. \\

\textbf{Bachelor of Science in Civil Engineering Students} &
Refers to the respondents of the study who are currently enrolled in or have previously completed the course Statics of Rigid Bodies in the Academic Year 2025--2026. These participants will provide data regarding their perception of the usability, accessibility, and satisfaction of the developed website. \\

\textbf{Bachelor of Science in Civil Engineering Instructors} &
Refers to the respondents of the study who are teaching the course Statics of Rigid Bodies during the Academic Year 2025--2026. Their feedback will contribute to evaluating the effectiveness, usability, and accessibility of the developed website as a supplementary learning tool. \\

\textbf{Digital Natives} &
Individuals who grew up in the age of digital technology and are comfortable using computers, the Internet, and web-based applications. They represent the target population of this study. \\

\textbf{Dynamics of Rigid Bodies} &
A branch of mechanics that examines the motion of non-deformable bodies under applied forces, including translational and rotational motion, using Newton’s laws and principles of energy and momentum. Prerequisite: Statics of Rigid Bodies. \\

\textbf{Free Body Diagram (FBD)} &
A Free Body Diagram (FBD) is a simplified graphical representation of a body or system isolated from its surroundings, showing all external forces and moments acting on it. It is used to visualize and solve problems involving equilibrium. \\

\textbf{Mechanics of Deformable Bodies} &
Also known as the mechanics of materials, this field studies the behavior of solid bodies that deform under applied forces. It analyzes how stresses and strains are distributed within materials and relates these deformations to material properties. Prerequisite: Statics of Rigid Bodies. \\

\textbf{Perception} &
The way students and instructors interpret and evaluate their experiences with a learning tool, concept, or environment. In this study, it refers to how they perceive the usability, accessibility, and effectiveness of the developed website. \\

\textbf{Statics of Rigid Bodies} &
A branch of engineering mechanics that studies forces and their effects on rigid bodies, focusing on conditions of equilibrium where the sum of all forces and moments equals zero. This course is the main focus of the study. \\

\textbf{Satisfaction} &
The degree to which Bachelor of Science in Civil Engineering students enrolled in Statics of Rigid Bodies and their instructors feel content or fulfilled with the developed website, \textit{StatiCalcs}, reflecting how well it meets their learning needs and expectations. \\

\textbf{Usability} &
The degree to which Bachelor of Science in Civil Engineering students enrolled in Statics of Rigid Bodies and their instructors can effectively, efficiently, and satisfactorily use the developed website, \textit{StatiCalcs}, to achieve their learning and teaching goals. \\

\end{longtblr}
