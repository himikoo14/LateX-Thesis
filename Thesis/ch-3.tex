\chapter{Methodology}\label{ch:3}

This chapter will present the research method, respondents, locale, instrument, and procedure to be used in the study. It will explain how the research will be conducted to determine the perceptions of Bachelor of Science in Civil Engineering students and instructors toward \textit{StatiCalcs}. This interactive web-based learning tool will be developed for the subject Statics of Rigid Bodies.

%---------------------------------------------------------------------

\section{Research Method}\label{sec:3-method}

    \hspace{1cm}This study will use a quantitative research method, which focuses on collecting and analyzing numerical data to describe patterns, relationships, and differences among variables. According to \cite{apuke_quantitative_2017}, quantitative research involves a systematic investigation that uses statistical techniques to produce objective and measurable results. This method is appropriate for the present study because it seeks to gather data from Civil Engineering students and instructors to assess and compare their perceptions of the developed web-based learning tool, \textit{StatiCalcs}.

    \hspace{1cm}Specifically, a correlational quantitative approach will be applied. As defined by Creswell, this approach examines the relationship or difference between two or more variables to determine whether a significant association exists. In this study, it will be used to determine the significant difference between the perceptions of students and instructors regarding the usability, accessibility, and satisfaction of \textit{StatiCalcs}. This method will allow the researchers to statistically analyze the degree to which these two groups differ in their perceptions, providing insights into the effectiveness and acceptance of the developed learning tool across users.

%---------------------------------------------------------------------

\section{Research Respondents}\label{sec:3-respo}

The respondents of this study will be selected Bachelor of Science in Civil Engineering students and teaching faculty from the College of Engineering at Mindanao State University -- General Santos (MSU--Gensan). The student respondents will include second-year Bachelor of Science in Civil Engineering students who are either currently enrolled in or have previously taken the course Statics of Rigid Bodies (ENS161). This selection ensures that participants have sufficient background and experience with the subject to provide reliable feedback on the developed web-based learning tool, \textit{StatiCalcs}. 

The faculty respondents, on the other hand, will consist of instructors who are currently teaching or have previously taught Statics of Rigid Bodies, as they can provide expert evaluation regarding the tool’s usability, accessibility, and relevance to the course content.

%---------------------------------------------------------------------

\section{Research Locale}\label{sec:3-locale}

\hspace{1cm}This study will be conducted at the College of Engineering, Mindanao State University – General Santos (MSU-Gensan), located in Fatima, General Santos City. The college offers various engineering programs, including Civil, Mechanical, Electrical, and Agricultural and Biosystems Engineering. The research will primarily focus on the H Building, where most classroom lectures, computer laboratory sessions, and faculty offices are situated. This location is ideal for the study since it houses both the student respondents taking Statics of Rigid Bodies and the faculty members teaching the subject.

%---------------------------------------------------------------------

\section{Research Procedure}\label{sec:3-proc}

%add here gina pa add ni sir kay ian

The researchers will conceptualize, design, and develop a web-based learning tool called \textit{StatiCalcs}, an interactive platform with integrated calculators specifically designed for the subject Statics of Rigid Bodies. The system will undergo evaluation and approval by experts in the field to ensure its accuracy, functionality, and relevance to the course. A researcher-made questionnaire will be developed to measure the level of perception of Civil Engineering students and instructors in terms of usability, accessibility, and satisfaction. This questionnaire will then be evaluated by academic experts in the field to ensure its validity and reliability before being used in the study.

Before data gathering, a formal letter of permission to conduct the study will be submitted to the Dean of the College of Engineering at Mindanao State University--General Santos. Respondents will first be allowed to explore and use the developed website, \textit{StatiCalcs}, before the data collection process. The researchers will administer the questionnaire to the selected respondents, composed of Bachelor of Science in Civil Engineering students and instructors. After data collection, the gathered responses will be encoded, organized, and subjected to statistical analysis.

%......

\subsection{Website Development Process}

\begin{figure}[H]
    \centering
    \includegraphics[width=0.3\textwidth]{assets/SystemFlow.jpg}
    \caption{System Flow Chart}
    \label{fig:system_flow_chart}
\end{figure}

The website was developed using a modern workflow, beginning with the selection of Java as the backend language due to its reliability, security, and strong support for building large, complex web applications. GitHub was used as the code repository to manage versions, collaborate, and integrate with deployment tools. For the frontend, Tailwind CSS was chosen to create a clean, responsive, and consistent user interface using its utility-first styling approach. ChatGPT assisted throughout the development process by helping with code structuring, debugging, and design decisions, improving efficiency and workflow. Netlify was used as the hosting and deployment platform. By connecting the GitHub repository to Netlify, every update pushed to GitHub was automatically built and deployed online, ensuring continuous delivery and easy accessibility of the website to users.



\subsection{Website Navigation and Functional Flow}

\begin{figure}[H]
    \centering
    \includegraphics[width=1.1\textwidth]{assets/concon.png}
    \caption{User journey flow chart}
    \label{fig:user_journey_flow}
\end{figure}


The website begins at the Landing Page, which acts as the main entry point for users. From this page, users are provided with navigational options to explore introductory sections such as About / Project Story, Meet the Developers, and Contact / Feedback. These sections help users understand the objectives of the project, know the creators behind the system, and provide a communication channel for inquiries or suggestions.

From the Landing Page, users encounter a decision point labeled Choose Path, where they decide whether to proceed to the Learning Section or directly access the Solver Tools.

If the user selects the Learning Path, they are directed to the Introduction / Theory page, which presents foundational concepts in statics. Following this, users proceed to Topic Pages (Chapter 1: Introduction to Statics), which include lessons on essential topics such as Equilibrium, Distributed Loads, Structures, and 2D/3D Concepts. After gaining theoretical knowledge, users move to the Learn How to Use Solvers section, which offers instructional guides on how to properly use the computational tools for structural analysis.

To further support learning, users may visit the Reference Page, which contains formula compilations, example problems, and relevant learning materials. An FAQ / Help section is also available to address common issues and user concerns. If additional support is needed, users may contact the developers directly through the Contact Page.

Alternatively, if the user chooses the direct problem-solving pathway, they proceed to a second decision point called Select Solver Type. The available solver tools are aligned with the chapters covered in the theoretical section, including:

\begin{itemize}
    \item Chapter 1: Introduction to Statics
    \item Chapter 2: Force Systems – 2D and 3D Resultant Force Calculator
    \item Chapter 3: Equilibrium – Equilibrium Calculator
    \item Chapter 4: Structures – Truss Analysis Calculator
    \item Chapter 5: Distributed Loads – Structural Analysis Calculator for Distributed Load Effects
\end{itemize}

After selecting a specific solver, the user proceeds to Run Analysis \& Get Results. Following computation, a View Results Summary page displays the output in a clear and structured format.

From this results page, users have three options:

\begin{enumerate}
    \item Generate and view Force Diagrams / Deflection Plot
    \item Download results as PDF or formatted report
    \item Modify input parameters and Recalculate to explore different solutions or optimize results
\end{enumerate}

This structured flow allows users to either learn fundamental engineering concepts before solving problems or directly perform structural analysis based on their needs. The design ensures flexibility, user engagement, and accessibility while supporting both educational and analytical purposes.

\subsection{Website System Architecture}

The \textit{Components} folder contains reusable elements of the website, specifically the Header and Footer, which help maintain consistent navigation and accessibility across all pages. The \textit{Header} file inside this folder includes the main navigation menu with the \textit{Home} button that redirects users to the homepage, along with a \textit{Topics} dropdown that provides quick access to various chapters and calculators such as: \\\textit{Chapter 1: Introduction to Statics}, \textit{Chapter 2: Force Systems with a 2D Resultant Solver}, \textit{Chapter 3: Equilibrium with an Equilibrium Solver}, \textit{Chapter 4: Structures with a Truss Calculator}, and \textit{Chapter 5: Distributed Loads}. This dropdown is designed to make browsing easier and more efficient for users. 

The header also contains the About page that presents the purpose and background of the website. Meanwhile, the Footer file contains additional navigation links including \textit{About}, \textit{References}, \textit{Contact}, and \textit{Developers}. The About button directs users to the website's About page; the References page provides the sources used in developing the calculators and learning materials; the Contact page allows users to reach the proponents of the thesis; and the Developers page displays information about the creators of the website. These components together ensure that users can easily access important sections and navigate the site smoothly.

The \textit{Calcs} folder contains the core Java code files that power the calculators integrated into the website. These files are responsible for performing the computational logic, processing user inputs, and generating the corresponding solutions based on engineering principles and formulas. Each calculator, such as those for force systems, equilibrium, truss analysis, and distributed loads, relies on the scripts stored in this folder to accurately compute results. This makes the Calcs folder an essential component of the website's functionality, as it handles the main calculation processes that support the learning and problem-solving features of the platform.

The \textit{App} folder contains all the front-end code for all website pages, styled using Tailwind CSS, which controls their visual layout, responsiveness, and overall user interface. Each page inside the App folder is responsible for either displaying calculators, such as for 2D Resultant, 3D Solver, Equilibrium, Distributed Loads, and Structural Analysis, or for providing supporting information, such as Introduction, References, About, Contact, and Developer details. The Tailwind CSS classes ensure that all these pages are consistently designed, user-friendly, and visually appealing, making navigation smooth and calculator usage clear and efficient. In summary, the App folder manages the appearance, layout, and behavior of both the calculator and informational pages from the user's perspective.


%---------------------------------------------------------------------

\section{Statistical Tools}\label{sec:3-stat}


A researcher-made questionnaire will be utilized in this study. It consists of three parts, each measured on a five-point Likert scale, to evaluate respondents’ perceptions of the developed web-based learning tool, \textit{StatiCalcs}, specifically in terms of usability, accessibility, and user satisfaction.

The items were adapted from previous studies and guided by the Technology Acceptance Model (TAM). Usability items were adapted from \cite{mathur_students_2011} and TAM ease-of-use constructs, accessibility items were adapted from \cite{pedraza_assessing_2025} and general e-learning accessibility indicators, and satisfaction items were adapted from both \cite{pedraza_assessing_2025} and TAM usefulness constructs. All items were modified to reflect the specific context of the \textit{StatiCalcs} platform.

Each part of the instrument consists of 10 items, resulting in a total of 30 items. Respondents will rate each statement on a scale from 1 (Strongly Disagree) to 5 (Strongly Agree), allowing for quantitative analysis of their perceptions regarding the usability, accessibility, and satisfaction of \textit{StatiCalcs}.

\begin{table}[h!]
\centering
\caption{Likert Scale for Respondent Perception}
\renewcommand{\arraystretch}{1.5} % 1.5 spacing for entire table
\begin{tabular}{|c|c|c|p{5cm}|}
\hline
\textbf{Scale} & \textbf{Range} & \textbf{Description} & \textbf{Interpretation} \\ \hline
5 & 4.21 -- 5.00 & Strongly Agree & The respondent fully agrees with the statement. \\ \hline
4 & 3.41 -- 4.20 & Agree & The respondent generally agrees with the statement. \\ \hline
3 & 2.61 -- 3.40 & Neutral & The respondent neither agrees nor disagrees with the statement. \\ \hline
2 & 1.81 -- 2.60 & Disagree & The respondent generally disagrees with the statement. \\ \hline
1 & 1.00 -- 1.80 & Strongly Disagree & The respondent fully disagrees with the statement. \\ \hline
\end{tabular}
\renewcommand{\arraystretch}{1} % reset spacing
\end{table}

\section{Statistical Treatment}\label{sec:3-stat}

This research will utilize weighted mean and frequency distribution to determine the level of perception of the respondents towards the developed web-based learning tool, \textit{StatiCalcs}, in terms of usability, accessibility, and user satisfaction. Each item in the questionnaire will be measured using a five-point Likert scale, where respondents indicate their level of agreement or perception. The weighted mean will be computed to quantify the overall perception of students and instructors, providing a clear and interpretable measure of how the tool is perceived across its different aspects. Frequency distribution will also be used to show how many respondents selected each response, offering a detailed view of the spread of opinions and highlighting areas of agreement or disagreement. By combining these two statistical treatments, the study will provide both a summarized and a detailed understanding of respondents’ perceptions.
