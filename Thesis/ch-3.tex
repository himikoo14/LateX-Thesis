\chapter{Methodology}\label{ch:3}

    \hspace{1cm}This chapter presents the research method, respondents, locale, instrument, and procedure used in the study. It explains how the research will be conducted to determine the perceptions of Civil Engineering students and faculty toward \textit{StatiCalc}, an interactive web-based learning tool developed for the subject Statics of Rigid Bodies.

\section{Research Method}\label{sec:3-method}

    \hspace{1cm}This study will use a quantitative research method, which focuses on collecting and analyzing numerical data to describe patterns, relationships, and differences among variables. According to Apuke (2017), quantitative research involves a systematic investigation that uses statistical techniques to produce objective and measurable results. This method is appropriate for the present study because it seeks to gather data from Civil Engineering students and instructors to assess and compare their perceptions of the developed web-based learning tool, \textit{StatiCalc}.

    \hspace{1cm}Specifically, a correlational quantitative approach will be applied. As defined by Creswell (2014), this approach examines the relationship or difference between two or more variables to determine whether a significant association exists. In this study, it will be used to determine the significant difference between the perceptions of students and instructors regarding the usability, accessibility, and satisfaction of \textit{StatiCalc}. This method will allow the researchers to statistically analyze the degree to which these two groups differ in their perceptions, providing insights into the effectiveness and acceptance of the developed learning tool across users.

\section{Research Respondents}\label{sec:3-respo}

\hspace{1cm}The respondents of this study will be selected Bachelor of Science in Civil Engineering students and teaching faculty from the College of Engineering at Mindanao State University – General Santos (MSU-Gensan). The student respondents will include second-year and higher-level Civil Engineering students who are either currently enrolled in or have already completed the course Statics of Rigid Bodies (ENS 161). This selection ensures that participants have sufficient background and experience with the subject, allowing them to provide reliable feedback on the developed web-based learning tool, \textit{StatiCalc}. The faculty respondents, on the other hand, will consist of instructors who are currently teaching or have previously taught Statics of Rigid Bodies, as they can provide expert evaluation regarding the tool’s usability, accessibility, and relevance to the course content.

\section{Research Locale}\label{sec:3-locale}

\hspace{1cm}This study will be conducted at the College of Engineering, Mindanao State University – General Santos (MSU-Gensan), located in Fatima, General Santos City. The college offers various engineering programs, including Civil, Mechanical, Electrical, and Agricultural and Biosystems Engineering. The research will primarily focus on the Engineering Building, where most classroom lectures, computer laboratory sessions, and faculty offices are situated. This location is ideal for the study since it houses both the student respondents taking Statics of Rigid Bodies and the faculty members teaching the subject.

\section{Research Procedure}\label{sec:3-proc}

\hspace{1cm}The researchers will conceptualize, design, and develop a web-based learning tool called \textit{StatiCalc}, an interactive platform with integrated calculators specifically designed for the subject Statics of Rigid Bodies. The system will undergo evaluation and approval by experts in the field to ensure its accuracy, functionality, and relevance to the course. A researcher-made questionnaire will then be developed to measure the level of perception of Civil Engineering students and instructors in terms of usability, accessibility, and satisfaction. Before data gathering, a formal letter of permission to conduct the study will be submitted to the Dean of the College of Engineering at Mindanao State University–General Santos.

\hspace{1cm}The researchers will administer the questionnaire to the selected respondents composed of Civil Engineering students and faculty members. After data collection, the gathered responses will be encoded, organized, and subjected to statistical analysis.

\section{Statistical Tools}\label{sec:3-stat}

\hspace{1cm}A researcher-made instrument will be utilized in conducting this study. The tool will employ a five-point Likert scale to measure the level of perception of both students and instructors regarding the developed web-based learning tool, \textit{StatiCalc}. The survey will assess their perceptions in terms of usability, accessibility, and satisfaction. Each item in the questionnaire will be rated on a scale ranging from 1 (Strongly Disagree) to 5 (Strongly Agree), allowing for quantitative analysis of the respondents’ perceptions toward the tool’s overall usability, accessibility, and user satisfaction.