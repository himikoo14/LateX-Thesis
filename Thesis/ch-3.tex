\chapter{Methodology}\label{ch:3}

This chapter presented the research method, respondents, locale, instrument, and procedure used in the study. It explained how the research was conducted to determine the perceptions of Bachelor of Science in Civil Engineering students and instructors toward \textit{StatiCalcs}. This interactive web-based learning tool was developed for the Statics of Rigid Bodies course.

%---------------------------------------------------------------------

\section{Research Method}\label{sec:3-method}

    \hspace{1cm}This study used a quantitative research method, which focused on collecting and analyzing numerical data to describe patterns, relationships, and differences among variables. According to \cite{apuke_quantitative_2017}, quantitative research involved a systematic investigation that used statistical techniques to produce objective and measurable results. This method was appropriate for the present study because it sought to gather data from Civil Engineering students and instructors to assess and compare their perceptions of the developed web-based learning tool, \textit{StatiCalcs}.

    \hspace{1cm}Specifically, a correlational quantitative approach was applied. As defined by Creswell, this approach examined the relationship or difference between two or more variables to determine whether a significant association existed. In this study, it was used to determine the significant difference between the perceptions of students and instructors regarding the usability, accessibility, and satisfaction of \textit{StatiCalcs}. This method allowed the researchers to statistically analyze the degree to which these two groups differed in their perceptions, providing insights into the effectiveness and acceptance of the developed learning tool across users.

%---------------------------------------------------------------------

\section{Research Respondents}\label{sec:3-respo}

\hspace{1cm}The respondents of this study were selected Bachelor of Science in Civil Engineering students and teaching faculty from the College of Engineering at Mindanao State University -- General Santos (MSU--Gensan). The student respondents included second-year Bachelor of Science in Civil Engineering students who were either currently enrolled in or had previously taken the course Statics of Rigid Bodies (ENS161). This selection ensured that participants had sufficient background and experience with the subject to provide reliable feedback on the developed web-based learning tool, \textit{StatiCalcs}.

\hspace{1cm}The faculty respondents, on the other hand, consisted of instructors who were currently teaching or had previously taught Statics of Rigid Bodies, as they could provide expert evaluation regarding the tool's usability, accessibility, and relevance to the course content.

\hspace{1cm}Purposive sampling was used to select students who had taken or were currently enrolled in Statics of Rigid Bodies, and instructors who had experience teaching the course.

%---------------------------------------------------------------------

\section{Research Locale}\label{sec:3-locale}

\hspace{1cm}This study was conducted at the College of Engineering, Mindanao State University -- General Santos (MSU--Gensan), located in Fatima, General Santos City. The college offered various engineering programs, including Civil, Mechanical, Electrical, and Agricultural and Biosystems Engineering. The research primarily focused on the H Building, where most classroom lectures, computer laboratory sessions, and faculty offices were situated. This location was ideal for the study since it housed both the student respondents taking Statics of Rigid Bodies and the faculty members teaching the subject.

%---------------------------------------------------------------------

\section{Research Procedure}\label{sec:3-proc}

%add here gina pa add ni sir kay ian

The researchers conceptualized, designed, and developed a web-based learning tool called \textit{StatiCalcs}, an interactive platform with integrated calculators specifically designed for the subject Statics of Rigid Bodies. The system underwent evaluation and approval by experts in the field to ensure its accuracy, functionality, and relevance to the course. 

A researcher-made questionnaire was developed to measure the level of perception of Civil Engineering students and instructors in terms of usability, accessibility, and satisfaction. This questionnaire was then evaluated by academic experts in the field to ensure its validity and reliability before being used in the study.

Before data gathering, a formal letter of permission to conduct the study was submitted to the Dean of the College of Engineering at Mindanao State University--General Santos. Respondents were first allowed to explore and use the website \textit{StatiCalcs} before the data collection process. They were given approximately fifteen (15) days to use and interact with the platform to ensure adequate exposure and familiarity with its features and tools before answering the survey. The researchers administered the questionnaire to the selected respondents, composed of Bachelor of Science in Civil Engineering students and instructors. After data collection, the gathered responses were encoded, organized, and subjected to statistical analysis.

Participation in the study was voluntary, and all responses were treated with strict confidentiality to ensure the privacy and protection of the respondents.

%......

\subsection{Website Development Process}

\begin{figure}[H]
    \centering
    \includegraphics[width=0.3\textwidth]{assets/SystemFlow.jpg}
    \caption{System Flow Chart}
    \label{fig:system_flow_chart}
\end{figure}

The website was developed using a modern workflow, beginning with the selection of Java as the backend language due to its reliability, security, and strong support for building large, complex web applications. GitHub was used as the code repository to manage versions, collaborate, and integrate with deployment tools. For the frontend, Tailwind CSS was chosen to create a clean, responsive, and consistent user interface using its utility-first styling approach. ChatGPT assisted throughout the development process by helping with code structuring, debugging, and design decisions, improving efficiency and workflow. Netlify was used as the hosting and deployment platform. By connecting the GitHub repository to Netlify, every update pushed to GitHub was automatically built and deployed online, ensuring continuous delivery and easy accessibility of the website to users.



\subsection{Website Navigation and Functional Flow}

\begin{figure}[H]
    \centering
    \includegraphics[width=1.1\textwidth]{assets/concon.png}
    \caption{User journey flow chart}
    \label{fig:user_journey_flow}
\end{figure}


The website began at the Landing Page, which acted as the main entry point for users. From this page, users were provided with navigational options to explore introductory sections such as About / Project Story, Meet the Developers, and Contact / Feedback. These sections helped users understand the objectives of the project, know the creators behind the system, and provide a communication channel for inquiries or suggestions. 

From the Landing Page, users encountered a decision point labeled Choose Path, where they decided whether to proceed to the Learning Section or directly access the Solver Tools. 

If the user selected the Learning Path, they were directed to the Introduction / Theory page, which presented foundational concepts in statics. Following this, users proceeded to Topic Pages (Chapter 1: Introduction to Statics), which included lessons on essential topics such as Equilibrium, Distributed Loads, Structures, and 2D/3D Concepts. After gaining theoretical knowledge, users moved to the Learn How to Use Solvers section, which offered instructional guides on how to properly use the computational tools for structural analysis.

To further support learning, users visited the Reference Page, which contained formula compilations, example problems, and relevant learning materials. An FAQ / Help section was also available to address common issues and user concerns. If additional support was needed, users contacted the developers directly through the Contact Page.

Alternatively, if the user chose the direct problem-solving pathway, they proceeded to a second decision point called Select Solver Type. The available solver tools were aligned with the chapters covered in the theoretical section, including:

\begin{itemize}
    \item Chapter 1: Introduction to Statics
    \item Chapter 2: Force Systems --- 2D and 3D Resultant Force Calculator
    \item Chapter 3: Equilibrium --- Equilibrium Calculator
    \item Chapter 4: Structures --- Truss Analysis Calculator
    \item Chapter 5: Distributed Loads --- Structural Analysis Calculator for Distributed Load Effects
\end{itemize}

After selecting a specific solver, the user proceeded to Run Analysis and Get Results. After the computation, a View Results Summary page displayed the output in a clear, structured format.

From this results page, users had three options:

\begin{enumerate}
    \item Generate and view Force Diagrams / Deflection Plot
    \item Download results as PDF or formatted report
    \item Modify input parameters and Recalculate to explore different solutions or optimize results
\end{enumerate}

This structured flow allows users to either learn fundamental engineering concepts before solving problems or directly perform structural analysis based on their needs. The design ensures flexibility, user engagement, and accessibility while supporting both educational and analytical purposes.

\subsection{Website System Architecture}

The \textit{Components} folder contained reusable elements of the website, specifically the Header and Footer, which helped maintain consistent navigation and accessibility across all pages. The \textit{Header} file inside this folder included the main navigation menu with the \textit{Home} button that redirected users to the homepage, along with a \textit{Topics} dropdown that provided quick access to various chapters and calculators such as: \\\textit{Chapter 1: Introduction to Statics}, \textit{Chapter 2: Force Systems with a 2D Resultant Solver}, \textit{Chapter 3: Equilibrium with an Equilibrium Solver}, \textit{Chapter 4: Structures with a Truss Calculator}, and \textit{Chapter 5: Distributed Loads}. This dropdown was designed to make browsing easier and more efficient for users.

The header also contained the About page that presented the purpose and background of the website. Meanwhile, the Footer file contained additional navigation links including \textit{About}, \textit{References}, \textit{Contact}, and \textit{Developers}. The About button directed users to the website's About page; the References page provided the sources used in developing the calculators and learning materials; the Contact page allowed users to reach the proponents of the thesis; and the Developers page displayed information about the creators of the website. These components together ensured that users could easily access important sections and navigate the site smoothly.

The \textit{Calcs} folder contained the core Java code files that powered the calculators integrated into the website. These files were responsible for performing the computational logic, processing user inputs, and generating the corresponding solutions based on engineering principles and formulas. Each calculator, such as those for force systems, equilibrium, truss analysis, and distributed loads, relied on the scripts stored in this folder to accurately compute results. This made the Calcs folder an essential component of the website's functionality, as it handled the main calculation processes that supported the learning and problem-solving features of the platform.

The \textit{App} folder contained all the front-end code for all website pages, styled using Tailwind CSS, which controlled their visual layout, responsiveness, and overall user interface. Each page inside the App folder was responsible for either displaying calculators, such as for 2D Resultant, 3D Solver, Equilibrium, Distributed Loads, and Structural Analysis, or for providing supporting information, such as Introduction, References, About, Contact, and Developer details. The Tailwind CSS classes ensured that all these pages were consistently designed, user-friendly, and visually appealing, making navigation smooth and calculator usage clear and efficient. In summary, the App folder managed the appearance, layout, and behavior of both the calculator and informational pages from the user's perspective.


%---------------------------------------------------------------------

\section{Statistical Tools}\label{sec:3-stat}


A researcher-made questionnaire was utilized in this study. It consisted of three parts, each measured on a five-point Likert scale, to evaluate respondents’ perceptions of the developed web-based learning tool, \textit{StatiCalcs}, specifically in terms of usability, accessibility, and user satisfaction.

The items were adapted from previous studies and guided by the Technology Acceptance Model (TAM). Usability items were adapted from \cite{mathur_students_2011} and TAM ease-of-use constructs, accessibility items were adapted from \cite{pedraza_assessing_2025} and general e-learning accessibility indicators, and satisfaction items were adapted from both \cite{pedraza_assessing_2025} and TAM usefulness constructs. All items were modified to reflect the specific context of the \textit{StatiCalcs} platform.

Each part of the instrument consisted of 10 items, resulting in a total of 30 items. Respondents rated each statement on a scale from 1 (Strongly Disagree) to 5 (Strongly Agree), allowing for quantitative analysis of their perceptions regarding the usability, accessibility, and satisfaction of \textit{StatiCalcs}.

To ensure the internal consistency and reliability of the researcher-made instrument, Cronbach's alpha was computed for each of the three sections: usability, accessibility, and satisfaction. The reliability testing confirmed that the questionnaire items were statistically consistent and appropriate for use in the study.

\begin{table}[h!]
\centering
\caption{Likert Scale for Respondent Perception}
\renewcommand{\arraystretch}{1.5} % 1.5 spacing for entire table
\begin{tabular}{|c|c|c|p{5cm}|}
\hline
\textbf{Scale} & \textbf{Range} & \textbf{Description} & \textbf{Interpretation} \\ \hline
5 & 4.21 -- 5.00 & Strongly Agree & The respondent fully agrees with the statement. \\ \hline
4 & 3.41 -- 4.20 & Agree & The respondent generally agrees with the statement. \\ \hline
3 & 2.61 -- 3.40 & Neutral & The respondent neither agrees nor disagrees with the statement. \\ \hline
2 & 1.81 -- 2.60 & Disagree & The respondent generally disagrees with the statement. \\ \hline
1 & 1.00 -- 1.80 & Strongly Disagree & The respondent fully disagrees with the statement. \\ \hline
\end{tabular}
\renewcommand{\arraystretch}{1} % reset spacing
\end{table}


%---------------------------------------------------------------------

\section{Statistical Treatment}\label{sec:3-stat}

This research utilized weighted mean and frequency distribution to determine the level of perception of the respondents toward the developed web-based learning tool, \textit{StatiCalcs}, in terms of usability, accessibility, and user satisfaction. Each item in the questionnaire was measured using a five-point Likert scale, where respondents indicated their level of agreement or perception. The weighted mean was computed to quantify the overall perception of students and instructors, providing a clear and interpretable measure of how the tool was perceived across its different aspects. Frequency distribution was also used to show how many respondents selected each response, offering a detailed view of the spread of opinions and highlighting areas of agreement or disagreement.

In addition, an independent samples t-test was conducted to determine whether there was a significant difference between the perceptions of students and instructors regarding the usability, accessibility, and satisfaction of the developed learning tool. By combining these statistical treatments, the study provided both a summarized and comprehensive understanding of respondents’ perceptions.
