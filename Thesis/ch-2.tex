\chapter{Review of Related Literature}

%\section{A Random Section}

%This is a citation using the MLA style %\autocite{gioncu_earthquake_2011}.

\section*{\textbf{Statics of Rigid Bodies}}
\hspace{1cm} \textit{Statics of Rigid Bodies}, commonly referred to as \textit{Engineering Statics}, is a branch of engineering mechanics that focuses on the study of forces, moments, and their effects on bodies that remain at rest or in equilibrium. It serves as one of the most fundamental subjects in engineering education, as it establishes the foundational principles necessary for advanced engineering courses.

\hspace{1cm} In the Civil Engineering curriculum of Mindanao State University -- General Santos (MSU-Gensan), as stated in MSU Board Resolution No. 375, s. 2017, \textit{Statics of Rigid Bodies (ENS 161)} serves as a core foundation course and a prerequisite to \textit{Dynamics of Rigid Bodies (ENS 162)} and \textit{Mechanics of Deformable Bodies (CVE 155)}, both of which are essential for advanced design and analysis. Similarly, in other engineering disciplines, such as Mechanical, Electrical, and Agricultural and Biosystems Engineering, \textit{Statics} also serves as a preparatory course that cultivates the analytical and problem-solving skills fundamental to engineering education. Failure in this subject often results in academic delays, as it serves as a prerequisite for subsequent major engineering courses. Overall, mastering \textit{Statics of Rigid Bodies} is essential not only for academic progression but also for developing the critical thinking and analytical reasoning required of future engineers.

\hspace{1cm} Beyond academics, the principles of \textit{Statics} are extensively applied in real-world engineering practice. They are used in designing safe and efficient systems such as buildings, bridges, dams, machines, and various mechanical components. Civil engineers use \textit{Statics} to ensure that structural loads are properly supported, mechanical engineers apply it to design stable machine elements, and electrical engineers use it to develop reliable mounting and support systems. Mastery of \textit{Statics} thus enables engineers to design structures and mechanisms that remain safe, stable, and functional under various loading conditions.

%---------------------------
%challenges
\section*{\textbf{Challenges in Learning Statics}}
\hspace{1cm} Learning \textit{Statics of Rigid Bodies} has been widely recognized as one of the most challenging areas in engineering education due to its high conceptual and analytical demands. Salami and Perry (2025) explored the perceptions of students regarding the difficulties they face in \textit{Statics} and identified a wide range of conceptual, procedural, and affective barriers. Students often struggled to connect mathematical symbols with their physical meanings, treating forces and moments as abstract quantities rather than as real interactions between bodies. Many had difficulty constructing \textit{Free Body Diagrams (FBDs)}, frequently misplacing force directions, omitting reactions, or confusing internal and external forces. They also expressed challenges in distinguishing between related concepts such as moments, couples, and resultants, and in visualizing force systems within two- or three-dimensional contexts. Beyond conceptual gaps, the study also highlighted emotional and instructional factors: students often described \textit{Statics} as a difficult and anxiety-inducing subject, with fast-paced lectures and heavy workloads compounding their struggles. The authors concluded that these issues call for more interactive, visualization-based, and student-centered learning environments to promote deeper comprehension and engagement.

\hspace{1cm} Complementing the students’ perspective, Salami, Oladipo, and Perry (2024) examined \textit{Statics} instruction from the viewpoint of faculty members and teaching assistants. Their findings revealed that educators recognized similar conceptual deficiencies among students, particularly in applying theoretical principles to problem-solving and in constructing logical reasoning processes. Faculty respondents attributed these difficulties to gaps in prerequisite mathematical knowledge, limited spatial visualization skills, and a tendency among students to rely heavily on memorization. Furthermore, instructors noted that traditional lecture-based instruction often fails to engage students or accommodate diverse learning styles. Faculty members also cited structural challenges such as large class sizes, limited time for individualized feedback, and insufficient resources for implementing interactive or simulation-based teaching methods. The study emphasized that overcoming these persistent learning barriers requires instructional innovations, such as technology-enhanced tools, simulations, and guided visualization platforms, that bridge the gap between abstract theory and practical understanding.

%______________
\section*{\textbf{Web-Based Learning}}

\hspace{1cm} In today’s generation, the internet has become widely accessible, allowing people to obtain information more easily than through traditional printed materials. Unlike books, online resources can be accessed anytime and anywhere, using various devices such as computers, tablets, or smartphones, as long as there is an internet connection. This accessibility has transformed how individuals learn and share knowledge. 

\hspace{1cm} \textit{Web-based learning}, also known as \textit{online learning} or \textit{e-learning}, utilizes internet-based platforms to deliver educational content and facilitate interaction between students and instructors. It provides learners with the flexibility to study at their own pace and convenience, overcoming barriers of time and location. Moreover, it encourages the use of multimedia tools—such as videos, simulations, and interactive modules—that enhance engagement and improve understanding of complex topics. As education continues to adapt to technological advancements, \textit{web-based learning} has become an essential component of modern instruction. It supports independent learning, promotes collaboration through virtual classrooms, and offers a more inclusive and adaptive learning environment for digital-native students who are already accustomed to using technology in their daily lives.

\hspace{1cm} According to Mishra \textit{et al.} (2022), the rise of \textit{e-learning} has greatly accelerated the global adoption of online learning, making it a dominant educational model in many countries. Their study revealed that as digital connectivity becomes more pervasive, \textit{e-learning}—particularly the emerging online learning systems—is expected to become the standard mode of instruction across multiple sectors. The widespread availability of web-based technologies enables a seamless, data-driven learning experience that supports continuous access to educational resources worldwide. As education continues to evolve with technological advancements, \textit{web-based learning} remains a vital component of modern instruction, supporting independent learning, promoting collaboration through virtual classrooms, and fostering an adaptive learning environment for students in the digital age.
%_--------------------
\section*{\textbf{StatiCalcs}}

\hspace{1cm} Söllradl (2022), in the study \textit{“Symbolic Calculation of Beam Structures: Digitalisation of Teaching Strength of Materials at University,”} conducted at the Institute for Applied Mechanics, TU Graz, discussed the digitalisation initiative at Graz Technical University (TU Graz) that aims to enhance digital learning in \textit{Engineering Mechanics} across several departments. The researcher developed an automated and scalable system designed to generate individualized assignments for students, supporting a more efficient and interactive approach to learning.

\hspace{1cm} To accomplish this, the researcher utilized a Python-based framework capable of automatically creating, distributing, and solving assignments related to beam structures. Instead of relying solely on numerical solutions, the system emphasized symbolic computation using the \textit{Euler–Bernoulli beam theory} to formulate and solve systems of linear equations. This method provided learners with a clearer understanding of the mechanical behavior of beam structures by reinforcing analytical and theoretical principles rather than focusing only on numerical outputs.

\hspace{1cm} The developed tool also allowed for the modeling of beams with varying flexural and axial rigidity, included stiff elements, and supported the use of different types of joints and bearings at the nodes. Through this approach, the researcher demonstrated how digitalization and automation can be effectively integrated into engineering education to promote personalized learning, minimize manual workload for instructors, and enhance students’ analytical and problem-solving skills.

\hspace{1cm} Gfrerer, Michael H., Benjamin Marussig, Katharina Maitz, and Mia M. Banger (2024), in their study titled \textit{“Teaching Mechanics with Individual Exercise Assignments and Automated Correction,”} introduced an automated correction system for exercise assignments designed to address the challenges of manually creating and grading student work in large engineering classes. Traditional assignment methods often limit the number of exercises that can be provided, resulting in a smaller problem pool relative to the number of students. This restriction makes it difficult to assess whether learners can independently solve problems and encourages unreflective task replication, which hinders conceptual understanding and leads to inaccurate self-assessment. 

\hspace{1cm} To overcome these limitations, the researchers developed a scalable framework for generating, distributing, and automatically correcting individualized exercises in topics such as \textit{statics, strength of materials, dynamics,} and \textit{hydrostatics}. Their system allows each student to receive a unique set of problems, promoting academic integrity and active engagement. A quantitative survey conducted among students enrolled in a statics course demonstrated strong acceptance of the tool, with feedback indicating that it enhanced self-directed and reflective learning. The authors concluded that the automated correction system provides significant added value in mechanics education by fostering independent learning and reducing instructor workload, thereby improving the overall efficiency and quality of teaching in engineering mechanics courses.

\hspace{1cm} Building upon these findings, the researchers developed \textit{StatiCalcs}, a web-based supplementary learning tool designed specifically for students taking \textit{Statics of Rigid Bodies}. The platform provides an interactive computational environment where users can perform statics-related analyses, visualize results, and verify manual solutions. \textit{StatiCalcs} is not intended to replace traditional instruction but to serve as a supplementary resource that reinforces theoretical and analytical concepts through immediate computational feedback and graphical representation.

\hspace{1cm} The structure and content of \textit{StatiCalcs} are based on key topics presented in \textit{Meriam and Kraige’s Engineering Mechanics: Statics, 5th Edition}, one of the most widely used references in engineering education. The system consists of four core chapters: \textit{Force Systems, Equilibrium, Structural Analysis,} and \textit{Distributed Loads}. The \textit{Force Systems} module enables users to determine the resultant of both two-dimensional and three-dimensional force systems using the tip-to-tail graphical method and the analytical component method. The \textit{Equilibrium} module facilitates the analysis of concurrent, parallel, and general force systems by applying the equations of equilibrium. Meanwhile, the \textit{Structural Analysis} module features a Truss Calculator that computes internal member forces using the method of joints and the method of sections, allowing students to verify their manual computations. Lastly, the \textit{Distributed Loads} module supports the analysis of beams and other statically determinate structures subjected to linearly or non-linearly varying loads.

\hspace{1cm} By combining computational accuracy, visual interpretation, and automation, \textit{StatiCalcs} provides a technologically enhanced platform that aligns with modern engineering education practices. As a supplementary educational tool, it bridges the gap between theoretical instruction and applied problem-solving by allowing learners to test, visualize, and confirm their understanding of fundamental statics principles. Ultimately, \textit{StatiCalcs} contributes to improving student engagement, conceptual comprehension, and analytical proficiency in the study of \textit{Statics of Rigid Bodies}.

\hspace{1cm} Several existing web-based calculators have been developed to support learning in mathematics and engineering. \textit{Symbolab}, for instance, is a widely used online computational platform that focuses primarily on algebraic manipulation, calculus, and introductory physics. It employs symbolic computation to provide step-by-step solutions, helping learners understand the procedural flow of mathematical problem-solving (Symbolab, 2024). Similarly, \textit{GeoGebra} offers a dynamic learning environment that integrates geometry, algebra, and graphing functionalities. Its interactive visualization capabilities make it particularly useful for exploring mathematical relationships and geometric principles in an intuitive manner (Hohenwarter \& Lavicza, 2022).

\hspace{1cm} Another notable platform is \textit{SkyCiv}, which serves as a specialized online tool for structural and civil engineering applications. \textit{SkyCiv} allows users to perform two-dimensional and three-dimensional analyses of beams, trusses, and frames, making it a practical choice for both professionals and students. However, its advanced features are typically accessible through paid subscriptions, which can limit accessibility for some learners (SkyCiv, 2024).

\hspace{1cm} In contrast to these existing platforms, \textit{StatiCalcs} was developed as a free, topic-specific web-based learning tool dedicated solely to the study of \textit{Statics of Rigid Bodies}. Unlike general-purpose platforms, \textit{StatiCalcs} focuses exclusively on the fundamental concepts of engineering mechanics, including \textit{Force Systems, Equilibrium, Structural Analysis,} and \textit{Distributed Loads}. This narrow yet specialized scope allows the system to provide a more structured, focused, and educationally aligned approach to problem-solving in statics.

\hspace{1cm} Furthermore, \textit{StatiCalcs} integrates both computational and visualization features tailored to the statics curriculum. Users can input problem parameters, perform step-by-step calculations, and visualize results such as force systems, truss configurations, and distributed load diagrams. These functions make the platform an effective supplementary resource for verifying manual solutions, reinforcing theoretical lessons, and enhancing student understanding of equilibrium and force interaction principles. 

\hspace{1cm} By focusing specifically on statics and maintaining open accessibility, \textit{StatiCalcs} addresses a gap in the availability of free, academically oriented web-based tools for engineering education. It provides a targeted, user-friendly, and pedagogically consistent alternative to general-purpose platforms such as \textit{Symbolab}, \textit{GeoGebra}, and \textit{SkyCiv}. Through its combination of computational precision, visual learning support, and subject-specific design, \textit{StatiCalcs} aligns closely with current trends in digital and technology-enhanced learning, contributing to the ongoing evolution of engineering education.

%------------
\section*{\textbf{Technology Based Learning}}

\hspace{1cm}Study of (De La Hoz, J. L., et al.) states how self-explanation activities may support student learning in statics. Specifically, this study examines the characteristics of student self-explanations of worked examples and their relationship with students' conceptual change. The findings suggest a relationship between the type of worked example, students' approaches to self-explaining, and their conceptual change and problem-solving skills in statics. To increase the quality of the students' explanations and to improve their conceptual understanding, additio

\hspace{1cm}El-Shaimaa Talaat Abumandour (2022), \textit{“Applying E-Learning System for Engineering Education – Challenges and Obstacles,”} emphasized the rapid technological advancements of the 21st century that have transformed educational delivery methods. The researcher noted that the continuous improvement of technology has led to the emergence of e-learning as a modern instructional approach widely adopted by educational institutions, public organizations, and academic libraries. The researcher highlighted that engineering education is increasingly moving toward a blended learning model, which effectively integrates traditional “Face to face” instruction with computer-assisted methodologies and internet-based learning. This hybrid approach enhances accessibility, flexibility, and engagement among learners.

\hspace{1cm}The study also outlined several challenges and obstacles that stakeholders, including teachers, professors, and librarians, must address to fully develop and sustain effective e-learning systems. These challenges include ensuring technological readiness, maintaining instructional quality, and supporting both instructors and students in adapting to digital learning environments. The researcher further proposed recommendations aimed at strengthening the connection between e-learning and engineering education, thereby promoting a more adaptive, innovative, and inclusive academic experience.

\hspace{1cm}This correlates with our Web-based learning method that utilizes the Internet as a platform for delivering educational content, simulations, and interactive experiences. It allows learners to access information and perform tasks beyond the limits of traditional classroom settings. This mode of learning promotes flexibility, self-paced study, and accessibility, which are highly beneficial in modern education. With the continuous advancement of technology and the widespread availability of the Internet, it has become increasingly important for educational institutions to adopt innovative tools and platforms that enhance the learning process. Integrating technology into education not only broadens access to knowledge but also encourages independent learning and practical application of theoretical concepts.

\hspace{1cm}The target population of this web-based learning tool are students currently enrolled in Statics of Rigid Bodies, typically second-year engineering students. This subject is a core component of the engineering curriculum, serving as the foundation for more advanced courses such as mechanics of materials, structural analysis, and dynamics. However, many students find statics challenging due to the complexity of force interactions and the abstract nature of equilibrium concepts. Through web-based tools, learners can engage with interactive calculators, visualizations, and simulations that reinforce computational techniques and problem-solving strategies related to statics.

\hspace{1cm}These students belong to the generation that has grown up alongside rapid technological advancement. Commonly referred to as digital natives, they are accustomed to using online platforms, mobile applications, and digital resources in their daily lives. This familiarity with technology makes web-based learning an appropriate and effective approach for their academic environment. By integrating educational content with interactive computation, web-based learning tools such as StatiCalcs align with the learning habits of modern students, providing a convenient, accessible, and engaging platform for applying engineering principles.

\hspace{1cm}Overall, web-based learning represents a significant shift in educational delivery, supporting the development of both technical competence and digital literacy among engineering students.