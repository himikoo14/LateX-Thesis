\chapter{Review of Related Literature}

The Review of Related Literature discusses previous studies and theories relevant to this research. It focuses on the challenges students face in learning Statics of Rigid Bodies, the use of web-based learning tools, and learners' perceptions of technology-assisted instruction. This review highlights the need for an interactive platform like \textit{StatiCalcs} to support effective learning and teaching.

%---------------------------------------------------------------------

\section{Statics of Rigid Bodies in Engineering Education}

Statics of Rigid Bodies is a fundamental area of Engineering Mechanics concerned with understanding how forces and moments act on bodies that remain in equilibrium. As a foundational subject, Statics develops the analytical reasoning necessary for advanced topics such as Dynamics, Strength of Materials (Mechanics), Structural Analysis, and Fluid Mechanics. 

In the Civil Engineering curriculum of Mindanao State University -- General Santos (MSU-Gensan), as stated in MSU Board Resolution No. 375, s. 2017, Statics of Rigid Bodies (ENS161) is formally designated as a foundation course and a prerequisite for higher engineering subjects. Its role as a gateway subject underscores the need for students to develop strong problem-solving, visualization, and mathematical reasoning skills before moving on to more complex engineering analyses. 

For Civil Engineering students, mastery of Statics is essential because it directly relates to designing structures that safely resist loads. Failure to grasp the principles of force systems, equilibrium, and moment calculations may hinder students’ academic progression and reduce their preparedness for professional engineering practice \autocite{salami_exploring_2024}. Its foundational role in engineering education highlights the need for effective teaching approaches and supportive learning resources.

%---------------------------------------------------------------------

\section{Conceptual Difficulties in Learning Statics}

Learning Statics poses substantial conceptual challenges for students, mainly because the subject requires interpreting forces and interactions that are not directly observable. Students frequently struggle to visualize force directions, understand moment effects, and connect theoretical principles to real-world mechanical behavior. These conceptual gaps often lead to incorrect assumptions about how forces act on objects and how equilibrium conditions are achieved.

One of the most common sources of difficulty is the abstraction involved in translating a real object into a simplified engineering model. Students must isolate bodies, idealize supports, and mentally detach systems from their environment, steps essential to constructing valid Free Body Diagrams (FBDs). When learners are unable to visualize these relationships clearly, their problem-solving accuracy decreases significantly.

Misconceptions about force components, resultant forces, and moment calculations further contribute to students’ struggles. Many learners also find it challenging to distinguish between couples, force systems, and distributed loads. These conceptual difficulties suggest the need for instructional approaches that enhance visualization, provide immediate feedback, and reduce students’ dependence on rote memorization.

%---------------------------------------------------------------------

\section{Procedural and Analytical Challenges}

In addition to conceptual issues, \cite{salami_challenges_2025} emphasized that students often face procedural difficulties when applying Statics principles to computational tasks. Many struggle to correctly break forces into components, apply equilibrium equations, or follow systematic steps when solving problems. These challenges frequently arise from gaps in prior mathematical knowledge, particularly in algebra and trigonometry.

Procedural errors also occur when students attempt to construct FBDs without fully understanding the physical systems involved. Incorrect or incomplete diagrams often lead to errors that persist through the entire solution process. According to \cite{salami_challenges_2025}, procedural weaknesses combined with conceptual misunderstandings contribute to persistent difficulties in Statics, even among higher-year students.

Students also tend to rely on memorized formulas without understanding their derivations or applications. This approach hampers their ability to transfer knowledge to unfamiliar problem types or adapt to more complex engineering tasks. Such procedural weaknesses highlight the need for supplementary learning resources that guide students step by step and reinforce systematic problem-solving habits.

%---------------------------------------------------------------------

\section{Mathematical Readiness in Statics}

A strong mathematical foundation is a prerequisite for success in Statics of Rigid Bodies. However, many incoming engineering students lack proficiency in algebra, geometry, trigonometry, and vector operations, skills essential for analyzing forces and moments. \cite{felix_analyzing_2023} reports that declining math preparedness among first-year students negatively affects performance in computation-heavy engineering subjects.

\cite{wenceslao_mathematical_2022} conducted a study assessing the mathematical and analytical readiness of incoming first-year engineering students in Eastern Visayas, Philippines. The study revealed that 57\% of students were not mathematically college-ready, demonstrating significant deficiencies in key areas such as calculus, trigonometry, and algebra, which are essential for problem-solving in engineering disciplines like Statics. Students struggled particularly with integration, derivatives, and trigonometric identities, indicating gaps in foundational skills needed to manipulate equations, solve simultaneous equations, resolve forces, and understand vector relationships. These weaknesses often lead to frustration, disengagement, and errors in calculating resultants, determining moment arms, and resolving angled forces.

The findings underscore the need for early interventions, remedial support, digital or supplementary learning tools that scaffold mathematical procedures, reducing the cognitive burden of complex computations and allowing learners to focus on conceptual understanding and practical application in Statics. Digital resources that automate or guide mathematical procedures may help students focus more on conceptual interpretation and problem comprehension.

%---------------------------------------------------------------------

\section{Math Anxiety and Its Impact on Engineering Students}

Math anxiety is a widely documented factor that affects students’ performance in technical subjects. \cite{incierto_mathematics_nodate} found that students with higher levels of anxiety tend to perform poorly in mathematics-related courses, including engineering subjects that rely heavily on quantitative skills. This emotional barrier often causes learners to avoid practicing mathematical problems, ultimately reducing their mastery of essential skills.

In Statics, where problem-solving requires applying multiple mathematical steps, anxiety can impair students’ ability to think clearly, interpret diagrams, and follow procedural logic. Reducing math anxiety is therefore an important aspect of improving student performance.

Several studies suggest that allowing students to use calculators, digital tools, or guided computational aids can mitigate anxiety by reducing the pressure associated with manual calculations. \cite{segarra_does_2024} found that structured calculator use improves students’ confidence and accuracy, enabling them to concentrate more on conceptual understanding than on arithmetic manipulation.

%---------------------------------------------------------------------

\section{Visualization and Spatial Ability in Engineering}

Spatial reasoning is a key competency in engineering, particularly in Statics, where students must interpret diagrams, imagine 3D force interactions, and understand geometric relationships. Learners with weak spatial ability often struggle with vector representation, force decomposition, and moment visualization \autocite{fontaine_correlating_2024}.

Visualization tools, including interactive diagrams and simulations, help develop these skills by allowing students to manipulate variables and observe changes in real time. \cite{zeichner_using_2020} argues that simulations enhance student motivation and conceptual understanding, particularly in visually demanding subjects such as Statics. Given that Statics requires a combination of spatial and analytical skills, web-based learning tools that incorporate visualization are essential for supporting diverse learning needs.

%---------------------------------------------------------------------

\section{Importance of Free Body Diagram Construction}

The Free Body Diagram (FBD) is one of the most critical tools in Engineering Mechanics. It serves as the foundation for analyzing force interactions, solving equilibrium equations, and understanding mechanical behavior. Despite its importance, many students struggle to construct accurate FBDs, often omitting forces, misrepresenting directions, or misidentifying support reactions \autocite{salami_challenges_2025}.

Errors in FBD construction typically result in incorrect equilibrium equations, making the entire solution process invalid. Researchers emphasize the need for learning tools that support FBD visualization, provide real-time diagrams, and reinforce correct diagramming techniques. Visual learning tools and graphical simulations have been shown to improve performance and deepen understanding in Statics.


%---------------------------------------------------------------------

\section{Technology-Based Learning for Engineering Student}

The concept of digital natives, introduced by Prensky, describes individuals who have grown up immersed in digital technology, such as computers, smartphones, and the internet. This group generally includes those born from the 1980s onward, commonly identified as Millennials, Generation Z, and Generation Alpha. These generations grew up during the rapid advancement of technology, which has profoundly influenced their lifestyles, behaviors, and learning preferences. Since they have never known a world without digital connectivity, technology has become an integral part of their identity and daily lives. According to \cite{szymkowiak_information_2021}, educators should combine traditional teaching methods with modern, internet-based learning tools, including mobile applications and online videos, to accommodate the learning preferences. \cite{zeichner_using_2020} similarly found that learning through simulation fosters a deeper understanding of abstract principles and concepts compared with conventional instruction. These findings are supported by \cite{de_la_hoz_self-explanation_2023}, who observed that traditional lecture-based teaching in statics often fails to address students’ learning barriers and problem-solving difficulties. 

Also, a comprehensive literature review by \cite{tawafak_innovative_2021} highlights that the continuous exposure of Gen Z to digital environments has transformed their learning preferences and expectations in higher education. Their familiarity with interactive, technology-driven systems opens opportunities for educators to design more engaging, adaptive digital learning platforms. Collectively, these studies emphasize the importance of integrating interactive, technology-enhanced approaches into higher education to meet the learning needs of digital natives in modern academic settings and to improve student engagement, comprehension, and analytical skills, especially in conceptually demanding courses such as Statics of Rigid Bodies.

Technology-based learning enhances engagement, improves retention, and provides flexibility for self-paced study. Web-based learning tools have been shown to promote active learning, encourage repeated practice, and support independent problem-solving. These tools can supplement traditional lectures, which often lack interactivity and visualization. Given these advantages, integrating technology into Statics instruction is essential to address students’ difficulties and enhance their learning experience.


%---------------------------------------------------------------------

\section{Web-Based Learning in Engineering Education}

Web-based learning has become a significant trend in modern education due to its accessibility, flexibility, and multimedia integration capabilities. Through internet-based platforms, students can access materials anytime, practice problems repeatedly, and receive immediate feedback. This is especially useful in technical subjects that require repeated exposure to problem-solving procedures.

The study of \cite{de_la_hoz_self-explanation_2023} states how self-explanation activities may support student learning in Statics. Specifically, this study examines the characteristics of student self-explanations of worked examples and their relationship with students' conceptual change. The findings suggest a relationship among the type of worked example, students' self-explanation approaches, and their conceptual change and problem-solving skills in Statics. To improve students' explanations and conceptual understanding, additional prompts or initial training in self-explanation may be required within the worked-examples context. The researchers suggest that interactive learning environments enhance conceptual understanding by prompting students to explain solutions and reflect on errors.
Research by \cite{abumandour_applying_2021} on the potential of e-learning as an educational system for engineering topics highlighted the rapid technological advancements of the 21st century, which have transformed educational delivery methods. The researcher noted that advances in technology have led to the emergence of e-learning as a modern instructional approach widely adopted by educational institutions, public organizations, and academic libraries. The researcher highlighted that engineering education is increasingly moving toward a blended learning model that effectively integrates traditional ``face-to-face'' instruction with computer-assisted methodologies and internet-based learning. This hybrid approach enhances accessibility, flexibility, and learner engagement.

The study also outlined several challenges and obstacles that stakeholders, including teachers, professors, and librarians, must address to develop and sustain effective e-learning systems fully. These challenges include ensuring technological readiness, maintaining instructional quality, and supporting both instructors and students in adapting to digital learning environments. The researcher further proposed recommendations to strengthen the connection between e-learning and engineering education, thereby promoting a more adaptive, innovative, and inclusive academic experience. The researcher highlights the growing need for digital tools in engineering education as institutions shift toward blended and online learning systems.

In the study of \cite{sollradl_symbolic_nodate}, which aims to enhance digital learning in Engineering Mechanics across several departments, the researcher developed an automated, scalable system to generate individualized assignments for students, supporting a more efficient, interactive approach to learning. To accomplish this, the researcher used a Python-based framework that automatically generates, distributes, and solves assignments for beam structures. Instead of relying solely on numerical solutions, the system emphasized symbolic computation using the \textit{Euler--Bernoulli beam theory} to formulate and solve linear systems of equations. This method provided learners with a clearer understanding of the mechanical behavior of beam structures by reinforcing analytical and theoretical principles rather than focusing only on numerical outputs. Through this approach, this study demonstrated how digitalization and automation can be effectively integrated into engineering education to promote personalized learning, reduce instructors' manual workload, and enhance students' analytical and problem-solving skills.

Another significant advancement in web-based learning was introduced by \cite{gfrerer_teaching_2023}, who developed an automated correction system for individualized exercise assignments in Engineering Mechanics courses. Their work addresses long-standing challenges in large classes, where manually creating and grading assignments limits the variety of problems students can encounter. With smaller problem pools, learners may replicate solutions without developing genuine conceptual understanding, and instructors may struggle to assess independent problem-solving skills.

To overcome these issues, the researchers developed a scalable digital framework capable of generating, distributing, and automatically correcting personalized exercises in Statics, Strength of Materials, Dynamics, and Hydrostatics. Each student receives a unique set of problems, promoting academic integrity and encouraging active, self-directed learning. A quantitative survey among Statics of Rigid Bodies students showed strong acceptance of the tool, with many reporting enhanced reflective learning and improved engagement. The authors concluded that automated correction systems significantly improve efficiency in mechanics education by reducing instructor workload while supporting deeper learner understanding.

Web-based tools allow for unified access across devices, making them ideal supplements for engineering subjects that require visual illustrations and computational support. These features align perfectly with the needs of Statics students who benefit from guided examples, diagrams, and automated computations.

%edit gfrgfr



%---------------------------------------------------------------------

\section{Related Systems and Existing Educational Tools}

Several digital tools and online platforms have been developed to support learning in mathematics and engineering, but each has limitations when applied explicitly to Statics of Rigid Bodies. Symbolab, for instance, provides automated, step-by-step solutions for algebra, calculus, and trigonometric problems. A study by \cite{paulin_effectiveness_2024} aimed to determine the effectiveness of the Symbolab Calculator in improving second-year students’ ability to solve trigonometric equations. The findings suggest that Symbolab can enhance students’ problem-solving skills and understanding of mathematical concepts. However, despite these benefits, Symbolab lacks dedicated features for statics topics such as equilibrium, force systems, and truss structures, limiting its direct applicability in engineering Statics courses.

GeoGebra offers dynamic and interactive visualizations for geometry and algebra. Its visual approach enhances students’ spatial reasoning and understanding of mathematical relationships. However, GeoGebra does not include built-in tools for engineering Statics, such as truss analysis or 2D/3D equilibrium calculations.

More specialized platforms, such as SkyCiv, provide advanced structural analysis tools for 2D and 3D frames and trusses. While highly useful for engineering applications, many features require paid subscriptions, reducing accessibility for undergraduate students.

Other digital innovations, such as the automated correction systems developed by \cite{gfrerer_teaching_2023}, focus on generating individualized problem sets and automated grading in Engineering Mechanics courses. These systems enhance self-directed learning but primarily support assessment rather than conceptual understanding. Similarly, symbolic computation tools developed by \cite{sollradl_symbolic_nodate} emphasize automated and symbolic solutions for beam structures in Strength of Materials, but do not provide comprehensive coverage of fundamental statics topics.

Although these tools contribute significantly to engineering education, none provide an integrated, free, and Statics-specific learning environment that combines step-by-step solutions, real-time computations, and dynamic visualizations. This gap underscores the need for a specialized platform like \textit{StatiCalcs}, designed for undergraduate Statics students, providing an accessible, interactive, and comprehensive learning tool that addresses both conceptual understanding and practical problem-solving.

%iedit paaaa

%---------------------------------------------------------------------

\section{Summary of Literature}

The literature consistently shows that Statics is conceptually demanding, procedurally complex, and dependent on strong mathematical foundations. Students struggle with visualization, FBD construction, force interactions, and equilibrium analysis. Traditional instruction alone often fails to address these learning gaps.

Research also emphasizes the importance of digital tools, visual learning, and web-based platforms in enhancing student understanding, especially for digital-native learners. Existing tools such as Symbolab, GeoGebra, and SkyCiv offer partial support but do not provide a comprehensive Statics learning environment.

The reviewed literature clearly indicates the need for an accessible, interactive, and specialized web-based tool to support students’ learning in Statics. This provides strong justification for developing \textit{StatiCalcs}, which integrates visualization, computation, and conceptual understanding to address the gaps identified in previous studies.